\documentclass[a4paper, 12pt]{report}		% general format

%%%% Charset
\usepackage{cmap}							% make PDF files searchable and copyable
\usepackage[utf8]{inputenc}					% accept different input encodings
\usepackage[T2A]{fontenc}					% russian font
\usepackage[russian]{babel}					% multilingual support (T2A)

%%%% Graphics
\usepackage[dvipsnames]{xcolor}			% driver-independent color extensions
\usepackage{graphicx}						% enhanced support for graphics
\usepackage{wrapfig}						% pro­duces fig­ures which text can flow around

%%%% Math
\usepackage{amsmath}						% Amer­i­can Math­e­mat­i­cal So­ci­ety (AMS) math fa­cil­i­ties
\usepackage{amsfonts}						% fonts from the AMS
\usepackage{amssymb}						% additional math symbols

%%%% Ty­po­grapy (don't forget about cm-super)
\usepackage{microtype}						% sublim­i­nal re­fine­ments to­wards ty­po­graph­i­cal per­fec­tion
\linespread{1.3}							% line spacing
\usepackage[left=2.5cm, right=1.5cm, top=2.5cm, bottom=2.5cm]{geometry}
\setlength{\parindent}{0pt}					% we don't want any paragraph indentation
\renewcommand{\chaptername}{}

%%%% Other
\usepackage{hyperref}							% ver­ba­tim with URL-sen­si­tive line breaks
%\DeclareUnicodeCharacter{00A0}{~}
\usepackage{float}

%------------------------------------------------------------------------------
\usepackage{listings}						% type­set source code list­ings

% Цвета для кода
\definecolor{string}{HTML}{101AF9}			% цвет строк в коде
\definecolor{comment}{HTML}{3F7F5F}		% цвет комментариев в коде
\definecolor{keyword}{HTML}{5F1441}		% цвет ключевых слов в коде
\definecolor{morecomment}{HTML}{8000FF}	% цвет include и других элементов в коде
\definecolor{captiontext}{HTML}{FFFFFF}	% цвет текста заголовка в коде
\definecolor{captionbk}{HTML}{999999}		% цвет фона заголовка в коде
\definecolor{bk}{HTML}{FFFFFF}				% цвет фона в коде
\definecolor{frame}{HTML}{999999}			% цвет рамки в коде

% Настройки отображения кода
\lstset{
	language=C++,							% Язык кода по умолчанию
	morekeywords={*,...},					% если хотите добавить ключевые слова, то добавляйте
	% Цвета
	keywordstyle=\color{keyword}\ttfamily\bfseries,
	stringstyle=\color{string}\ttfamily,
	commentstyle=\color{comment}\ttfamily\itshape,
	morecomment=[l][\color{morecomment}]{\#},
	% Настройки отображения
	breaklines=true,						% Перенос длинных строк
	basicstyle=\ttfamily\footnotesize,		% Шрифт для отображения кода
	backgroundcolor=\color{bk},				% Цвет фона кода
	%frame=lrb,xleftmargin=\fboxsep,xrightmargin=-\fboxsep, % Рамка, подогнанная к заголовку
	frame=tblr								% draw a frame at all sides of the code block
	rulecolor=\color{frame},				% Цвет рамки
	tabsize=2,								% tab space width
	showstringspaces=false,					% don't mark spaces in strings
	% Настройка отображения номеров строк. Если не нужно, то удалите весь блок
	numbers=left,							% Слева отображаются номера строк
	stepnumber=1,							% Каждую строку нумеровать
	numbersep=5pt,							% Отступ от кода
	numberstyle=\small\color{black},		% Стиль написания номеров строк
	% Для отображения русского языка
	extendedchars=true,
    literate=
        {Ö}{{\"O}}1                    {Ä}{{\"A}}1                    {Ü}{{\"U}}1
        {ß}{{\ss}}1                    {ü}{{\"u}}1                    {ä}{{\"a}}1
        {ö}{{\"o}}1                    {~}{{\textasciitilde}}1        {а}{{\selectfont\char224}}1
        {б}{{\selectfont\char225}}1    {в}{{\selectfont\char226}}1    {г}{{\selectfont\char227}}1
        {д}{{\selectfont\char228}}1    {е}{{\selectfont\char229}}1    {ё}{{\"e}}1
        {ж}{{\selectfont\char230}}1    {з}{{\selectfont\char231}}1    {и}{{\selectfont\char232}}1
        {й}{{\selectfont\char233}}1    {к}{{\selectfont\char234}}1    {л}{{\selectfont\char235}}1
        {м}{{\selectfont\char236}}1    {н}{{\selectfont\char237}}1    {о}{{\selectfont\char238}}1
        {п}{{\selectfont\char239}}1    {р}{{\selectfont\char240}}1    {с}{{\selectfont\char241}}1
        {т}{{\selectfont\char242}}1    {у}{{\selectfont\char243}}1    {ф}{{\selectfont\char244}}1
        {х}{{\selectfont\char245}}1    {ц}{{\selectfont\char246}}1    {ч}{{\selectfont\char247}}1
        {ш}{{\selectfont\char248}}1    {щ}{{\selectfont\char249}}1    {ъ}{{\selectfont\char250}}1
        {ы}{{\selectfont\char251}}1    {ь}{{\selectfont\char252}}1    {э}{{\selectfont\char253}}1
        {ю}{{\selectfont\char254}}1    {я}{{\selectfont\char255}}1    {А}{{\selectfont\char192}}1
        {Б}{{\selectfont\char193}}1    {В}{{\selectfont\char194}}1    {Г}{{\selectfont\char195}}1
        {Д}{{\selectfont\char196}}1    {Е}{{\selectfont\char197}}1    {Ё}{{\"E}}1
        {Ж}{{\selectfont\char198}}1    {З}{{\selectfont\char199}}1    {И}{{\selectfont\char200}}1
        {Й}{{\selectfont\char201}}1    {К}{{\selectfont\char202}}1    {Л}{{\selectfont\char203}}1
        {М}{{\selectfont\char204}}1    {Н}{{\selectfont\char205}}1    {О}{{\selectfont\char206}}1
        {П}{{\selectfont\char207}}1    {Р}{{\selectfont\char208}}1    {С}{{\selectfont\char209}}1
        {Т}{{\selectfont\char210}}1    {У}{{\selectfont\char211}}1    {Ф}{{\selectfont\char212}}1
        {Х}{{\selectfont\char213}}1    {Ц}{{\selectfont\char214}}1    {Ч}{{\selectfont\char215}}1
        {Ш}{{\selectfont\char216}}1    {Щ}{{\selectfont\char217}}1    {Ъ}{{\selectfont\char218}}1
        {Ы}{{\selectfont\char219}}1    {Ь}{{\selectfont\char220}}1    {Э}{{\selectfont\char221}}1
        {Ю}{{\selectfont\char222}}1    {Я}{{\selectfont\char223}}1    {і}{{\selectfont\char105}}1
        {ї}{{\selectfont\char168}}1    {є}{{\selectfont\char185}}1    {ґ}{{\selectfont\char160}}1
        {І}{{\selectfont\char73}}1     {Ї}{{\selectfont\char136}}1    {Є}{{\selectfont\char153}}1
        {Ґ}{{\selectfont\char128}}1
}

% Для настройки заголовка кода
\usepackage{caption}
\DeclareCaptionFont{white}{\color{сaptiontext}}
\DeclareCaptionFormat{listing}{\parbox{\linewidth}{\colorbox{сaptionbk}{\parbox{\linewidth}{#1#2#3}}\vskip-4pt}}
%\captionsetup[lstlisting]{format=listing,labelfont=white,textfont=white}
\renewcommand{\lstlistingname}{Листинг} % Переименование Listings в нужное именование структуры
\setlength{\parskip}{0.5cm}

\usepackage{pgfplots}
\pgfplotsset{compat=1.9}

%------------------------------------------------------------------------------
\begin{document}

\input{titlepage}                            % inclide the title page
\tableofcontents
%\input{text}                                 % inclide the main text
%\input{sources}                              % inclide the list of sources

\chapter*{Задание 1. Поиск узкоспециализированных источников.}
\addcontentsline{toc}{chapter}{Задание 1. Поиск узкоспециализированных источников.}

\textbf{Задание:} Поиск узкоспециализированных источников по теме – сайты и работы отдельных научных коллективов:

\begin{itemize}
\item НИИ;
\item научные лаборатории;
\item научные группы;
\item отдельные ученые.
\end{itemize}

Представить по каждому подпункту по 5-7 наиболее релевантных источников.

\vspace{1cm}

Санкт-Петербургский государственный университет: Центр технологий распределенных
реестров СПбГУ (\url{https://dltc.spbu.ru/}). Осуществляет исследования, разрабатывают образовательные программы, оказываются экспертные услуги и создают программные решения с использованием технологий распределенных реестров (блокчейн).

Федеральное государственное автономное образовательное учреждение высшего образования «Пермский государственный национальный исследовательский университет» (ПГНИУ): Лаборатория криптоэкономики и блокчейн-систем (\url{http://www.psu.ru/fakultety/ekonomicheskij-fakultet/laboratorii/laboratoriya-kriptoekonomiki-i-blokchejn-sistem}). Исследования направлены на создание математических моделей и практическое применение блокчейн-систем и смарт-контрактов для решения прикладных задач, таких как межбанковские и биржевые расчеты, финансовые расчеты в международных холдингах и группах, открытое электронное голосование, подтверждение авторских прав на цифровой контент и электронный нотариат.

Федеральное государственное образовательное бюджетное учреждение высшего образования "Финансовый университет при Правительстве Российской Федерации": Блокчейн-лаборатория (\url{http://www.fa.ru/org/science/irce/blockchainlab/Pages/Home.aspx}). Финансовый университет выиграл и выполнил тендер Государственной думы РФ на проведение экспертно-аналитического исследования по теме, связанной с технологией блокчейна и криптовалютам. Проводит Научно-исследовательские и опытно-конструкторские работы, связанные с внедрением технологии блокчейн в государственном и корпоративном секторах.

Калифорнийский университет в Беркли (The University of California, Berkeley): Harness the power of blockchain and cryptocurrencies (\url{https://www.edx.org/professional-certificate/uc-berkeleyx-blockchain-fundamentals})

MIT (Массачусетский технологический институт): Blockchain Technologies: Business Innovation and Application (\url{https://executive.mit.edu/course/blockchain-technologies/a056g00000URaa7AAD.html})

Stanford School of Engineering: Cryptocurrencies and Blockchain Technologies (\url{https://online.stanford.edu/courses/xcs251-cryptocurrencies-and-blockchain-technologies}).

\begin{itemize}
\item Корхов Владимир Владиславович (Доктор экономических наук, Профессор кафедры управления рисками и страхования СПбГУ).
\item Сергей Владимирович Ивлиев (Кандидат экономических наук, Руководитель Лаборатории криптоэкономики и блокчейн-систем ПГНИУ)
\item Виталий Бутерин (Создатель Etherium)
\item Горгадзе Владимир Вячеславович (Ph.D. теоретической физики, Калифорнийский университет в Беркли)
\item Баргер Артем Иосифович (Научный сотрудник IBM, работающий в группе Cloud Foundation в Хайфской исследовательской лаборатории)
\end{itemize}

\chapter*{Задание 2. Представить полный список отобранных источников.}
\addcontentsline{toc}{chapter}{Задание 2. Представить полный список отобранных источников.}

\textbf{Задание:} Представить полный список отобранных источников (использовать результаты ЛР1 И ЛР2 и п.1 данной ЛР3), распределив их по пунктам

Списки находятся в файлах \textit{library.eng.bib} и \textit{library.rus.bib}

При классификации материалов, я максимально старался избегать повторов. Но это оказалось довольно затруднительной задачей, т.к. в одной работе могут встречаться и исторические аспекты и современное состояние.


\section*{2.1. Актуальность поставленной проблемы}
\addcontentsline{toc}{section}{2.1. Актуальность поставленной проблемы}

\begin{itemize}
\item Анализ блокчейн-технологии: основы архитектуры, примеры использования, перспективы развития, проблемы и недостатки
\item Технологии, реализующие принцип распределенного реестра, и возможности их применения в информационных системах
\end{itemize}

\section*{2.2. Формулировка сути проблемы}
\addcontentsline{toc}{section}{2.2. Формулировка сути проблемы}

\begin{itemize}
\item The essence of cryptocurrencies: Descriptive and comparative analysis
\item Технологические принципы функционирования блокчейн
\item Анализ блокчейн-технологии: основы архитектуры, примеры использования, перспективы развития, проблемы и недостатки
\end{itemize}

\section*{2.3. Объект и предмет исследований и разработок}
\addcontentsline{toc}{section}{2.3. Объект и предмет исследований и разработок}

\begin{itemize}
\item Технологические принципы функционирования блокчейн
\item Анализ блокчейн-технологии: основы архитектуры, примеры использования, перспективы развития, проблемы и недостатки
\end{itemize}

\section*{2.4. Соотношение темы с другими смежными темами и ее особенности, значимость}
\addcontentsline{toc}{section}{2.4. Соотношение темы с другими смежными темами и ее особенности, значимость}

\begin{itemize}
\item К вопросу о выборе платформы распределенного реестра при проектировании информационных систем финансового сектора экономики
\item Концепция распределенного реестра на основе графов в системах распределенной энергетики
\item Применение системы распределенного реестра для повышения оперативного учета и контроля технического состояния элементов распределенной информационной системы
\item Разработка элементов распределенной библиотеки на базе распределенного реестра для систем мониторинга и диагностики
\item Разработка модели распределенного хранилища данных на базе технологии распределенного реестра для систем мониторинга
\end{itemize}


\section*{2.5. Исторический аспект появления и развития рассматриваемой проблемы}
\addcontentsline{toc}{section}{2.5. Исторический аспект появления и развития рассматриваемой проблемы}

\begin{itemize}
\item Технологические принципы функционирования блокчейн
\item Анализ блокчейн-технологии: основы архитектуры, примеры использования, перспективы развития, проблемы и недостатки
\end{itemize}

\section*{2.6. Современное состояние проблемы}
\addcontentsline{toc}{section}{2.6. Современное состояние проблемы}

\begin{itemize}
\item Анализ практической реализации технологии распределённого реестра
\end{itemize}

\section*{2.7. Кто занимается данной проблемой}
\addcontentsline{toc}{section}{2.7. Кто занимается данной проблемой}

Изучением, развитием и практическим применением блокчейна занимается огромное количество как коммерческих компаний (это было видно при просмотре патентов, выданых в Китае), так и научных организаций (в задании 1 текущей работы несколько примеров из ведущих ВУЗов), и даже независимые сообщеста разработчиков.

\section*{2.8. Основные теоретические и методические положения}
\addcontentsline{toc}{section}{2.8. Основные теоретические и методические положения}

\begin{itemize}
\item Методика обеспечения своевременности и полноты обмена информационными ресурсами в корпоративных сетях с распределенным реестром
\item Анализ открытого блокчейна в рамках комплексной классификации технологии распределенного реестра
\item Анализ практической реализации технологии распределённого реестра
\item Proof-of-randomness protocol for blockchain consensus: the white paper version 1.0
\end{itemize}

\section*{2.9. Подходы (методы и методологии) к решению проблемы с программной или аппаратной реализацией}
\addcontentsline{toc}{section}{2.9. Подходы (методы и методологии) к решению проблемы с программной или аппаратной реализацией}

\begin{itemize}
\item Decision Making using the Blockchain Proof of Authority Consensus
\item Proof of Learning (PoLe): Empowering neural network training with consensus building on blockchains
\item Consensus mechanism for software-defined blockchain in internet of things
\item Weighted Byzantine Fault Tolerance consensus algorithm for enhancing consortium blockchain efficiency and security
\end{itemize}

\section*{2.10. Области применения результатов разрешения проблемы}
\addcontentsline{toc}{section}{2.10. Области применения результатов разрешения проблемы}

\begin{itemize}
\item Практическое применение технологий распределенного реестра
\item Blockchain Platform for Industrial Internet of Things
\item Blockchain-based Database to Ensure Data Integrity in Cloud Computing Environments
\end{itemize}

\section*{2.11. Выбор одного из рассмотренных подходов и поиск необходимых библиотек, платформ и фреймворков для моделирования.}
\addcontentsline{toc}{section}{2.11. Выбор одного из рассмотренных подходов и поиск необходимых библиотек, платформ и фреймворков для моделирования.}

В качестве интересного практического примера можно взять блокчейн Etherium: этой осенью они осуществили переход от одного алгоритма консесуса к другому.


\chapter*{Задание 3. Дополнить список цифровых источников.}
\addcontentsline{toc}{chapter}{Задание 3. Дополнить список цифровых источников.}

\textbf{Задание:} Дополнить список цифровых источников ЛР1, ЛР2 и ЛР3 своими 2-3 дополнительными ресурсами. При необходимости провести в них поиск источников.

\vspace{1cm}

Хорошим примером перехода от одной модели консенсуса к другой (от proof-of-work к proof-of-stake) является сеть Etherium. Дополнительным цифровым источником будет являться исходный код ноды блокчейна, размещённый в хранилище кода GitHub:\\
\url{https://github.com/ethereum/go-ethereum}.

Кроме того, имеется отдельный репозиторий, который содержит предложения пользователей по улучшению продукта. Эти приложения формально не являются научными документами, однако обладают некоторыми научными признаками:
\begin{itemize}
\item Они оформлены по устоявшемуся шаблону
\item Они проходят публичное обсуждение
\item Они подкреплены ссылками на другие значимые источники
\end{itemize}
Ссылка на репозиторий: \url{https://github.com/ethereum/EIPs}


\chapter*{Задание 4. Поиск источников с подходящими БД.}
\addcontentsline{toc}{chapter}{Задание 4. Поиск источников с подходящими БД.}

\textbf{Задание:} Предварительный поиск источников с подходящими БД с входными примерами для тестирования и проверки работоспособности выбранного подхода (п.2.11) по теме аналитического отчета, включая проверку наличия необходимых данных в существующих и подходящих по смыслу для исследования подхода:

\begin{enumerate}
\item{ Для задач кластеризации\\
\url{https://www.uni-marburg.de/fb12/arbeitsgruppen/datenbionik/data?language_sync=1} (с подробным описанием)\\
\url{https://www.uni-marburg.de/fb12/arbeitsgruppen/datenbionik/Daten?language_sync=1} (удобные архивы)\\
\url{http://cs.uef.fi/sipu/datasets/}\\
\url{http://ana.cachopo.org/datasets-for-single-label-text-categorization} (категоризация текстов)
}
\item{ Для задач классификации\\
\url{https://sci2s.ugr.es/keel/category.php?cat=clas&order=ins#sub2}
}
\item{ Базы тестовых и реальных данных разного происхождения (медицинские, геологические и т.п.)\\
\url{https://archive.ics.uci.edu/ml/index.php}
}
\item{ Для разных задач (нужно выбрать тип входных данных) - классификация (тексты, изображения), сегментация (изображения), аннотирование (документов)...\\
\url{https://dataturks.com/projects/trending}
}
\item{Сборники Top Sources For Machine Learning Datasets и 10 Standard Datasets for Practicing Applied Machine Learning\\
\url{https://towardsdatascience.com/top-sources-for-machine-learning-datasets-bb6d0dc3378b}
\url{https://machinelearningmastery.com/standard-machine-learning-datasets/}
}
\item{\url{https://www.openml.org/}}
\item{\url{https://data.world/datasets/classification}}

\end{enumerate}

Дополнить 2-3 дополнительными ресурсами с базами тестовых примеров (например, из соревнований по машинному обучению, хакатонов, и т.д.)

Если мы говорим про Etherium, то достаточно запустить ноду блокчейна, и в процессе синхронизации будут получены все данные, которые можно изучать. В то же время, объём этих данные просто огромен, по этой причине специалисты подготовили облегченные датасеты, которые удобно использовать в экспериментах.

\textbf{Блокчейн Ethereum (март 2019)}
Полные исторические данные о блокчейне Ethereum в реальном времени (BigQuery).

Датасет содержит 8 760 000 строк записей и подготовлен для экспериментов на платформе Kaggle.

\url{https://www.kaggle.com/datasets/bigquery/ethereum-blockchain}

\textbf{Публичный датасет для изучения смарт-контрактов (август 2018)}

Датасет был подготовлен в рамках демонстрации аналитических возможностей платформы гугл по анализу данных.

\url{https://bigquery.cloud.google.com/dataset/bigquery-public-data:ethereum_blockchain}

\textbf{Ethereum (2015)}

Маленький (килобайты) по объёму датасет с примерами торговли в течение дня.

\url{https://datahub.io/cryptocurrency/ethereum#data}

Также дополнительным источником датасетов могут служить платформы:
\begin{itemize}
\item https://data.world/datasets/ethereum
\item https://datahub.io/
\end{itemize}

\chapter*{Заключение}
\addcontentsline{toc}{chapter}{Заключение}

Было обнаружено достаточно большое количество узкоспециализированных источников -- лабораторий и кафедр при ведущих мировых ВУЗах. Таким образом, можно выделить следующие группы, занимающиеся изучением Блокчейн:
\begin{itemize}
\item Научное сообщество
\item Коммерческие компании
\item Независимые исследователи на GitHub
\end{itemize}

Кроме хранения исходного кода, GitHub выступает как коммуникационная платформа, где независимые исследователи представляют свои предложения для обсуждения сообществом.

При классификации собранных материалов наибольшую сложность вызвало определение подходящей категории для какой-либо публикации. Зачастую, одна и та же работа могла удовлетворять требованиям различных категорий, т.к. покрывала поставленные там вопросы.

В качестве одного из подходов к реализации блокчейна и используемых алгоритмов консенсуса мы упомянули Etherium. Этой осенью проект перешёл с proof-of-work на proof-of-stake подход.

Для анализа данных в Etherium можно установить ноду, и получить все операции которые когда-либо происходили в этой сети с момента её запуска. Однако сообщество заранее подготовило минималистичные датасеты, на которых можно проверить свои модели до запуска на больших данных.


\end{document}

%------------------------------------------------------------------------------
% Examples:
%
%
%
% \begin{figure}[H]
% \centering
% \includegraphics[scale=0.8]{res/pic01}
% \caption{Picture description}
% \end{figure}
%
%
%
% \begin{table}[htb]
%     \begin{tabularx}{\textwidth}{|X|c|c|c|c|c|}
%     \hline
%     \multirow{2}{*}{tb1} & \multirow{2}{*}{tbl2} & \multicolumn{4}{c|}{tbl3} \\
%     \cline{3-6}
%     {} & {} & A & B & C & D\\
%     \hline
%     Text & {} & Text & {} & {} & {} \\
%     \hline
%     \end{tabularx}
% \caption{Table description}
% \end{table}